\subsection{Construction de l'ensemble stable maximal}\label{mis}
La première étape de l'algorithme consiste en la construction du MIS dont la définition est la suivante et qui possède des propriétés intéressantes pour la suite :
\begin{defn}Soit $G=(V,E)$ un graphe, l'ensemble stable maximal est le plus grand sous-ensemble $S\subseteq V$ tel que le sous-graphe $G[S]$ induit par $S$ ne contient pas d'arêtes.\end{defn}

\begin{lemma}\label{lmmis1}
Dans tout graphe géométrique, la taille de ses ensembles stables maximaux est majorée par $3.8opt+1.2$ où $opt$ est la taille de l'ensemble connexe dominant minimum.
\end{lemma}

\begin{lemma}\label{lmmis2}
Toute paire de sous-ensembles complémentaires du MIS a exactement une distance de deux sauts.
\end{lemma}

\cref{lmmis1} garantit que le ratio de la solution de cet algorithme par rapport à la solution optimale est bien $4.8 + ln5$. La validité de ce lemme dépend beaucoup de nos choix d'implémentation pour les algorithmes de construction MIS. Ces algorithmes étant prévus pour fonctionner de façon distribuée, nos implémentations, non distribuées, ne se comporte donc pas exactement comme ceux-ci.

\cref{lmmis2} est d'une importance majeure pour la validité de la solution. En effet, si une paire de sous-ensembles complémentaires du MIS a une distance inférieure à deux sauts, le MIS sera plus grand que prévu. On risque donc dans ce cas d'obtenir un résultat final plus grand.
À l'inverse, si la distance entre deux sous ensembles complémentaires du MIS est de plus de deux sauts. L'algorithme, essayant de relier ces sous-ensembles du MIS en colorant des nœuds contenant au minimum un voisin dans chacun de ces sous ensembles, ne pourra pas relier ces deux sous-ensembles. L'ensemble obtenu sera constitué d'au moins 2 composantes connexes correspondant aux deux ensembles complémentaires du MIS ayant une distance de plus de deux sauts.

\paragraph{}
Afin de construire un tel MIS qui répond correctement à ces propriété nous nous sommes appuyés une des références \cite{cardei2002connected} cités par les auteurs de l'article. Comme précédemment dit, nous avons dû adapter l'algorithme pour un fonctionnement dans une architecture non distribuée qui faisait passer des messages entre les différents nœuds afin de connaître l'état du réseau. Dans notre cas, nous nous sommes servis d'un système de marquage pour connaître l'avancement de la construction du MIS mais le principe de l'algorithme reste inchangé :
\begin{itemize}
\item \verb?noir? : nœud dominant, appartient au MIS
\item \verb?gris? : nœud dominé, voisin d'un nœud dominant, n'appartient pas au MIS
\item \verb?blanc? : nœud encore non traité par l'algorithme
\item \verb?blanc actif? : état particulier d'un nœud potentiellement prêt à devenir dominant
\end{itemize}

\paragraph{}
Pour illustrer le fonctionnement de notre algorithme basé sur \cite{cardei2002connected} nous allons prendre l'exemple ci-dessous :
\begin{figure}
    \begin{minipage}[c]{.46\linewidth}
        \includegraphics{images/mis1.jpg}
        \caption{MIS : état initial}
        \label{mis1}
    \end{minipage} \hfill
    \begin{minipage}[c]{.46\linewidth}
        \includegraphics{images/mis2.jpg}
    \caption{MIS : leader}
    \label{mis2}
    \end{minipage}
\end{figure}

\begin{figure}
    \begin{minipage}[c]{.46\linewidth}
        \includegraphics{images/mis3.jpg}
        \caption{MIS : domination}
        \label{mis3}
    \end{minipage} \hfill
    \begin{minipage}[c]{.46\linewidth}
        \includegraphics{images/mis4.jpg}
    \caption{MIS : élection}
    \label{mis4}
    \end{minipage}
\end{figure}

\begin{figure}
    \begin{minipage}[c]{.46\linewidth}
        \includegraphics{images/mis5.jpg}
        \caption{MIS : domination}
        \label{mis5}
    \end{minipage} \hfill
    \begin{minipage}[c]{.46\linewidth}
        \includegraphics{images/mis6.jpg}
    \caption{MIS : état final}
    \label{mis6}
    \end{minipage}
\end{figure}

\newpage
L'algorithme commence par choisir un tout premier nœud comme leader, en l'occurrence nous avons fait le choix de prendre le sommet de plus haut degré ce qui semble être pour nous un choix justifié puisque une plus large zone du graphe peut être couvert par ce seul point (figure \ref{mis2}). Ce nœud hôte est ainsi coloré en noir pour marqué son appartenance au MIS et tous ses voisins en gris pour indiquer leur domination par ce dernier. On liste ensuite les voisins des tous les nœud gris et les marque comme \textit{actifs} ceci afin de choisir un nouveau nœud dominant (figure \ref{mis3}). Le prochain nœud à marqué en noir choisi parmi cette liste de \textit{blancs actifs}, celui qui possède le plus grand nombre de voisins blancs \textit{non actifs} devient dominateur (figure \ref{mis4}). Le reste de la liste des \textit{blancs actifs} est maintenu et mis à jour dans la suite de l'algorithme. Puisqu'un nouveau nœud noir a été choisi, on peut répéter l'algorithme jusqu'à avoir marqué tous les nœud ou que la liste des \textit{blancs actifs} soit vide (figure \ref{mis5}). Enfin, il reste plus qu'à renvoyer la liste de tous les nœuds en noir pour renvoyer l'ensemble des points qui compose le MIS (figure \ref{mis6}).

\paragraph{}
Il va ensuite s'agir d'étudier la complexité de cet algorithme. On suppose que le graphe est correctement construit et que nous avons facilement accès aux voisins d'un sommet donné, ce qui est le cas. L'algorithme parcourt simplement les points d'une liste qu'on maintient au fur et à mesure que nous marquons les points d'une certaine couleur; cette même liste à mise à jour en parcourant les voisins du point qui est actuellement traité. L'algorithme s'arrête soit parce que la liste que l'on a construit est vide soit parce qu'il n'y a plus de points du graphe à traiter (ils ont tous été colorés). Cela nous donne une majoration sur la complexité de l'algorithme qui est en $O(\Delta \times n)$, avec $\Delta$ le degré maximal du graphe et $n$ le nombre de sommets, ce qui correspond bien à un parcours de l'ensemble des points dans lequel on parcourt les voisins.