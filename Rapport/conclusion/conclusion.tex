\section{Conclusion}
Finalement, il s'avère que l'algorithme de Toussaint est moins efficace que l'algorithme de Ritter, que ce soit en terme de temps d'exécution, d'efficacité en tant que conteneur ou même de complexité mémoire. Cependant, comme dit précédemment, c'est un algorithme qui à l'air de bien se prêter à la parallélisation. Une des raisons qui pourrait pousser à préférer l'algorithme de Toussaint à celui de Ritter serait la volonté d'obtenir un conteneur de forme rectangulaire plutôt que circulaire pour être plus fidèle à un problème posé. En effet, tout dépend de la forme globale de la répartition des points et, plus généralement, du contexte du problème considéré.

Une dernière question que l'on pourrait se poser serait celle de la validité de ces algorithmes en 3 dimensions. Fondamentalement, cela ne change pas grand chose pour l'algorithme de Ritter, l'idée générale reste la même et il suffit d'adapter les formules pour un espace de dimension 3. En revanche, cela risque d'être plus compliqué pour le rectangle qui deviendrait alors un parallélépipède. Joseph O'Rourke a proposé une solution en 1985 dont la complexité est cubique, ce qui est bien supérieur à la complexité pseudo-linéaire de la version 2D. L'idée étant que 2 des 6 faces du parallélépipède doit être coplanaire avec 2 des arêtes de l'enveloppe convexe.