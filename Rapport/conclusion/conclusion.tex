Finalement, il s'avère que l'algorithme Li et al. est très intéressant d'un point de vue temps d'exécution. En effet, l'algorithme glouton qui est proposé est de complexité linéaire ce qui assure de trouver une solution en un temps séduisant. Cependant, bien que l'on puisse penser à une modeste qualité venant de la part d'un algorithme glouton, on nous assure un ratio de $(4.8+ln5)opt+1.2$ avec $opt$ la solution optimale, ce qui est remarquable. Nous avons donc ici, un compromis plus qu'intéressant entre temps de calcul et approximation de la meilleure solution.

On peut se poser une question sur cet algorithme, que les auteurs n'ont pas abordé : la pertinence d'un \textit{local searching}. En effet, il est coutume d'améliorer une solution donnée par une approche gloutonne au moyen de nombreuses micro optimisations que l'on appelle \textit{local searching}. On pourrait par exemple, prendre $n$ points de notre solution et tenter de les substituer par $n-1$ points tout en conservant les propriétés d'un ensemble dominant connexe.