\documentclass[a4paper,oneside,10pt]{article}
\usepackage[utf8]{inputenc}
\usepackage[T1]{fontenc}

\usepackage{makeidx}
\usepackage{graphicx}
\usepackage[french]{babel}
\usepackage{amsmath}
\usepackage{amssymb}
\usepackage{mathrsfs}
\usepackage{lmodern}
\usepackage{listings}
\usepackage{caption}
\usepackage{pgfplots}
\usepackage{hyperref}
\usepackage{url}
\usepackage{csquotes}
\usepackage{mathabx}
\usepackage{fancyhdr}
\pagestyle{fancy}
\fancyhead[L]{\thepage}
\fancyhead[R]{Ensemble Dominant Connexe Glouton}
\cfoot{}

\usepackage{afterpage}
\newcommand\blankpage{%
    \null
    \thispagestyle{empty}%
    \addtocounter{page}{-1}%
    \newpage
}

\pgfplotsset{compat=1.14}
\makeindex
\begin{document}
\title{Algorithme Li et al. : ensemble dominant connexe glouton}
\author{Alexandre Fernandez \& Sylvain Ung}
\maketitle
\thispagestyle{empty}

\hrulefill
\vspace*{1cm}
\begin{center}
\includegraphics[scale=0.15]{images/logo_upmc.jpg}%
\end{center}
\vspace*{1cm}
\hrulefill

\begin{center}\bfseries\Huge
  Rapport de Recherche
\end{center}
\afterpage{\blankpage}

\newpage
\thispagestyle{empty}
\tableofcontents
\newpage
\vspace*{1cm}
\begin{center}\bfseries\Large
Algorithme Li et al. : ensemble dominant connexe glouton
\end{center}

\vspace*{1cm}
\paragraph{}
\textbf{Résumé :} Il va s'agir dans ce rapport de recherche d'étudier une implémentation efficace d'un algorithme répondant au problème de l'ensemble dominant connexe qui est très largement utilisé pour modéliser un réseau sans-fil. L'algorithme étudié est celui de Li et al \cite{li2005greedy} qui propose une approche gloutonne de la solution. L'objectif étant de minimiser cet ensemble afin de réduire, en pratique, les coûts de maintenance par exemple.
\paragraph{}
\textbf{Mots-clés :} Théorie des graphes, ensemble dominant, ensemble stable maximum, connexe, NP-complet, algorithme glouton, Li et al.

\newpage
\printindex
\section{Introduction}
Etant donné un ensemble $\mathcal{S}$ de $n$ points sur $\mathbb R^2$, le rectangle minimum $\mathcal{R}$ est le plus petit rectangle orienté tel que $\forall p \in \mathcal{S}, p \in \mathcal{R}$. Il est évident que l'enveloppe convexe de $\mathcal{S}$ est contenu dans le rectangle minimum, on rappelle que l'enveloppe convexe $\mathcal{E}$ d'un ensemble de points  est le plus petit ensemble convexe qui contient ces points.

Ainsi, l'algorithme de Toussaint pour obtenir le rectangle minimum d'un nuage de points démarre ses calculs à partir d'une enveloppe convexe qu'on aura calculée au préalable. Pour ce faire, il existe plusieurs algorithmes permettant d'obtenir l'enveloppe convexe d'un nuage de points dont les complexités varient selon le cas de figure :
\begin{itemize}
\item Parcours de Graham : $\mathcal{O}(n\log n)$
\item Marche de Jarvis : $\mathcal{O}(nm)$ avec $m$ le nombre de sommets de l'enveloppe convexe ou $\mathcal{O}(n^2)$ dans le pire cas
\item Quickhull : $\mathcal{O}(n\log n)$ ou $\mathcal{O}(n^2)$ dans le pire cas mais en pratique cet algorithme est très efficace et la plus utilisée
\end{itemize}

Dans notre démarche, nous nous servirons du parcours de Graham pour calculer l'enveloppe convexe avant d'appliquer l'algorithme de Toussaint à proprement dit. Dans un deuxième temps, nous renouvellerons l'expérience avec l'algorithme de Ritter calculant le cercle minimum. Les résultats obtenus seront ensuite analysés pour déterminer la qualité en tant que conteneur de ces 2 algorithmes, les détails de l'étude seront montrés dans la partie \ref{efficacite} de ce rapport.

Enfin, nous nous poserons la question de la validité de ces algorithmes sur un espace de dimension 3.
\newpage
\section{Algorithme}
\subsection{Construction de l'ensemble stable maximal}\label{mis}
La première étape de l'algorithme consiste en la construction du MIS dont la définition est la suivante et qui possède des propriétés intéressantes pour la suite :
\begin{defn}Soit $G=(V,E)$ un graphe, l'ensemble stable maximal est le plus grand sous-ensemble $S\subseteq V$ tel que le sous-graphe $G[S]$ induit par $S$ ne contient pas d'arêtes.\end{defn}

\begin{lemma}\label{lmmis1}
Dans tout graphe géométrique, la taille de ses ensembles stables maximaux est majorée par $3.8opt+1.2$ où $opt$ est la taille de l'ensemble connexe dominant minimum.
\end{lemma}

\begin{lemma}\label{lmmis2}
Toute paire de sous-ensembles complémentaires du MIS a exactement une distance de deux sauts.
\end{lemma}

\cref{lmmis1} garantit que le ratio de la solution de cet algorithme par rapport à la solution optimale est bien $4.8 + ln5$. La validité de ce lemme dépend beaucoup de nos choix d'implémentation pour les algorithmes de construction MIS. Ces algorithmes étant prévus pour fonctionner de façon distribuée, nos implémentations, non distribuées, ne se comporte donc pas exactement comme ceux-ci.

\cref{lmmis2} est d'une importance majeure pour la validité de la solution. En effet, si une paire de sous-ensembles complémentaires du MIS a une distance inférieure à deux sauts, le MIS sera plus grand que prévu. On risque donc dans ce cas d'obtenir un résultat final plus grand.
À l'inverse, si la distance entre deux sous ensembles complémentaires du MIS est de plus de deux sauts. L'algorithme, essayant de relier ces sous-ensembles du MIS en colorant des nœuds contenant au minimum un voisin dans chacun de ces sous ensembles, ne pourra pas relier ces deux sous-ensembles. L'ensemble obtenu sera constitué d'au moins 2 composantes connexes correspondant aux deux ensembles complémentaires du MIS ayant une distance de plus de deux sauts.

\paragraph{}
Afin de construire un tel MIS qui répond correctement à ces propriété nous nous sommes appuyés une des références \cite{cardei2002connected} cités par les auteurs de l'article. Comme précédemment dit, nous avons dû adapter l'algorithme pour un fonctionnement dans une architecture non distribuée qui faisait passer des messages entre les différents nœuds afin de connaître l'état du réseau. Dans notre cas, nous nous sommes servis d'un système de marquage pour connaître l'avancement de la construction du MIS mais le principe de l'algorithme reste inchangé :
\begin{itemize}
\item \verb?noir? : nœud dominant, appartient au MIS
\item \verb?gris? : nœud dominé, voisin d'un nœud dominant, n'appartient pas au MIS
\item \verb?blanc? : nœud encore non traité par l'algorithme
\item \verb?blanc actif? : état particulier d'un nœud potentiellement prêt à devenir dominant
\end{itemize}

\paragraph{}
Pour illustrer le fonctionnement de notre algorithme basé sur \cite{cardei2002connected} nous allons prendre l'exemple ci-dessous :
\begin{figure}
    \begin{minipage}[c]{.46\linewidth}
        \includegraphics{images/mis1.jpg}
        \caption{MIS : état initial}
        \label{mis1}
    \end{minipage} \hfill
    \begin{minipage}[c]{.46\linewidth}
        \includegraphics{images/mis2.jpg}
    \caption{MIS : leader}
    \label{mis2}
    \end{minipage}
\end{figure}

\begin{figure}
    \begin{minipage}[c]{.46\linewidth}
        \includegraphics{images/mis3.jpg}
        \caption{MIS : domination}
        \label{mis3}
    \end{minipage} \hfill
    \begin{minipage}[c]{.46\linewidth}
        \includegraphics{images/mis4.jpg}
    \caption{MIS : élection}
    \label{mis4}
    \end{minipage}
\end{figure}

\begin{figure}
    \begin{minipage}[c]{.46\linewidth}
        \includegraphics{images/mis5.jpg}
        \caption{MIS : domination}
        \label{mis5}
    \end{minipage} \hfill
    \begin{minipage}[c]{.46\linewidth}
        \includegraphics{images/mis6.jpg}
    \caption{MIS : état final}
    \label{mis6}
    \end{minipage}
\end{figure}

\newpage
L'algorithme commence par choisir un tout premier nœud comme leader, en l'occurrence nous avons fait le choix de prendre le sommet de plus haut degré ce qui semble être pour nous un choix justifié puisque une plus large zone du graphe peut être couvert par ce seul point (figure \ref{mis2}). Ce nœud hôte est ainsi coloré en noir pour marqué son appartenance au MIS et tous ses voisins en gris pour indiquer leur domination par ce dernier. On liste ensuite les voisins des tous les nœud gris et les marque comme \textit{actifs} ceci afin de choisir un nouveau nœud dominant (figure \ref{mis3}). Le prochain nœud à marqué en noir choisi parmi cette liste de \textit{blancs actifs}, celui qui possède le plus grand nombre de voisins blancs \textit{non actifs} devient dominateur (figure \ref{mis4}). Le reste de la liste des \textit{blancs actifs} est maintenu et mis à jour dans la suite de l'algorithme. Puisqu'un nouveau nœud noir a été choisi, on peut répéter l'algorithme jusqu'à avoir marqué tous les nœud ou que la liste des \textit{blancs actifs} soit vide (figure \ref{mis5}). Enfin, il reste plus qu'à renvoyer la liste de tous les nœuds en noir pour renvoyer l'ensemble des points qui compose le MIS (figure \ref{mis6}).

\paragraph{}
Il va ensuite s'agir d'étudier la complexité de cet algorithme. On suppose que le graphe est correctement construit et que nous avons facilement accès aux voisins d'un sommet donné, ce qui est le cas. L'algorithme parcourt simplement les points d'une liste qu'on maintient au fur et à mesure que nous marquons les points d'une certaine couleur; cette même liste à mise à jour en parcourant les voisins du point qui est actuellement traité. L'algorithme s'arrête soit parce que la liste que l'on a construit est vide soit parce qu'il n'y a plus de points du graphe à traiter (ils ont tous été colorés). Cela nous donne une majoration sur la complexité de l'algorithme qui est en $O(\Delta \times n)$, avec $\Delta$ le degré maximal du graphe et $n$ le nombre de sommets, ce qui correspond bien à un parcours de l'ensemble des points dans lequel on parcourt les voisins.
\subsection{Construction de l'ensemble stable maximal : 2\up{ème} méthode}
Un autre algorithme pour le MIS a été proposé par les auteurs. C'est un algorithme distribué qui se résume en les étapes suivantes :
\begin{itemize}
\item Construire un arbre couvrant quelconque du graphe (par le biais de messages)
\item Choisir une racine et stocker la distance à la racine (le rang) dans chaque point (par le biais de messages)
\item Colorer la racine en noir et envoyer un message \verb?noir? a tout ses voisins dans le graphe (broadcast)
\item Lorsqu'un point reçoit un message \verb?noir?, il change sa couleur en gris et envoie un message \verb?gris? contenant son rang à tout ses voisins dans le graphe (broadcast)
\item Lorsqu'un point reçoit un message \verb?gris?, il vérifie si le message gris vient d'un point au-dessus de lui (de rang inférieur) dans l'arbre, dans ce cas il décrémente un compteur \verb?k? (initialisé a son nombre de voisins dans le graphe de rang inférieur dans l'arbre). Lorsque ce compteur arrive à zéro il devient noir et diffuse un message \verb?noir?.
\item Des envois de messages supplémentaires permettent à l'algorithme distribué de notifier la racine de la fin de la procédure.
\end{itemize}

\begin{figure}[ht]
\begin{center}
\includegraphics[scale=0.6]{images/figureMIS.jpg}
\caption{Construction du MIS}
\end{center}
\end{figure}

\paragraph{}
Notre implémentation consiste à traduire cet algorithme distribué en un algorithme classique :
\begin{itemize}
\item Pour l'arbre couvrant : une implémentation inspirée de Kruskal sans prendre en compte la longueur des segments et en utilisant des \textit{disjoint set}
\item Une racine arbitraire est choisie et un parcours en profondeur de l'arbre est réalisé pour initialiser les rangs
\item Ensuite un parcours en largeur de l'arbre reproduira le mieux le comportement de l'algorithme distribué :
	\begin{itemize}
	\item On ajoute la racine dans une file puis tant que la file est non vide :
	\item On dépile la liste et on parcours les voisins du nœud dépilé
	\item On cherche un voisin noir : dans ce cas on change la couleur du nœud actuel en gris et on ajoute ses fils dans l'arbre dans la file
	\item Sinon on compte les voisins gris qui ont un rang inférieur au nœud actuel, on change sa couleur en noir et on ajoute ses fils dans l'arbre dans la file 
	\end{itemize}
\item On retourne la liste des nœuds noirs
\end{itemize}

\paragraph{}
Il s'agit maintenant de donner une borne a la complexité de cet implémentation.
La construction de l'arbre couvrant consiste à parcourir tous les points du graphe. Pour chaque voisins n'étant pas dans le même d\textit{Disjoint-Set} (opération \verb?FIND?), faire l'union de leur \textit{disjoint set} avec celui du point considéré et construire la structure d'arbre (stocker les voisins du point dans l'arbre).
La complexité amortie des \textit{Disjoint-Set} optimisés peut être considérée comme constante. On réalise pour chaque point du graphe de degré $d$, au maximum $d$ \verb?find? et $d$ \verb?union?.

Soit $\Delta$ le degré maximal de notre graphe et $n$ le nombre de points. On est donc en $O(n \times \Delta)$.

L'initialisation des rang est un parcours en profondeur de l'arbre et dépend de l'étape précédente.
Le parcours en largeur effectué lors du reste de l'algorithme dépend aussi de la taille de l'arbre, chaque nœud a au maximum $\Delta-1$ fils et il y a $n$ nœuds.
La complexité de cet algorithme est donc en $O(n \times \Delta)$.

\newpage
\subsection{Construction de l'ensemble dominant connexe}
\newpage
\section{Résultats}
\subsection{Tests}

\paragraph{}
Pour tester notre algorithme nous avons eu besoin de deux choses :
\begin{itemize}
\item une fonction permettant de valider notre solution
\item un fonction permettant de générer aléatoirement des tests
\end{itemize}

\paragraph{}
La méthode \verb?isValid()? qui vérifie la validité de la solution calculée par notre algorithme est donnée par le pseudo-code suivant :
\begin{lstlisting}
isValid(ArrayList<Point> graph, ArrayList<Points> cds, int edgeTreshold) {
	valid = true;
	//build the graph structure of the cds
	ArrayList<NodeVertexDS> graphcds = graph(cds);
	
	//compute connex components
	for( v in graphcds ) {
		for( vn in v.neighbors ) {
			v.disjointsetelement.union(vn.disjointsetelement)
		}
	}
	
	compconnex = new HashSet<DisjointSetElement>();
	for( v in graphcds ) { compconnex.add(v.disjointsetelement.find()); }
	
	if(compconnex.size() > 1) { print("Error on connexity : " + compconnex.size()); valid = false) }
	
	rest = points.clone();
	rest.removeall(cds);
	
	//remove all neighbors of elements of cds from rest
	removeneighbors(rest, cds, edgeTreshold);
	
	if(rest.size() > 0) { print("Error dominating : " + rest.size()); valid = false; }
	
	return valid;
}
\end{lstlisting}

\paragraph{}
Nous avons aussi eu besoin de générer des tests. Pour cela il convient de générer des graphes géométriques de différentes tailles et avec des seuils différents (valeur maximale pour que 2 points soient considérés voisins l'un de l'autre).
Voici le pseudo code du générateur :
\begin{lstlisting}
generateGraph(width, heigth, nb, edgeTreshold) :
	result : ArrayList<Point>;
	while result.size() != nb :
		add random points in result until result.size() == nb
		compute connected components of result
		if there is multiple connected components :
			keep only the biggest connected components in points (remove the points that are in smaller components)
\end{lstlisting}

\paragraph{}
Nous avons ensuite établi plusieurs bases de test :

\begin{figure}[ht]
\begin{center}
\begin{tabular}{|*{3}{c|}}
    \hline
     Nombre de points  & Largeur $\times$ Hauteur  & Seuil \\
    \hline
    100  & 1000 $\times$ 1000 & 50 \\
    \hline
    500  & 1000 $\times$ 1000  & 50 \\
    \hline
    1000  & 1000 $\times$ 1000  & 50 \\
    \hline
    5000  & 1000 $\times$ 1000  & 50 \\
    \hline
    10000  & 1000 $\times$ 1000  & 50 \\
    \hline
\end{tabular}
\end{center}
\captionof{table}{Base de test 1}
\end{figure}

\begin{figure}[ht]
\begin{center}
\begin{tabular}{|*{3}{c|}}
    \hline
     Nombre de points  & Largeur $\times$ Hauteur  & Seuil \\
    \hline
    100  & 1000 $\times$ 1000 & 5 \\
    \hline
    500  & 500 $\times$ 500  & 25 \\
    \hline
    1000  & 1000 $\times$ 1000  & 50 \\
    \hline
    5000  & 5000 $\times$ 5000  & 250 \\
    \hline
    10000  & 10000 $\times$ 10000  & 500 \\
    \hline
\end{tabular}
\end{center}
\captionof{table}{Base de test 2}
\end{figure}

\begin{figure}[ht]
\begin{center}
\begin{tabular}{|*{3}{c|}}
    \hline
     Nombre de points  & Largeur $\times$ Hauteur  & Seuil \\
    \hline
    100  & 100 $\times$ 100 & 25 \\
    \hline
    500  & 500 $\times$ 500  & 36 \\
    \hline
    1000  & 1000 $\times$ 1000  & 50 \\
    \hline
    5000  & 5000 $\times$ 5000  & 161 \\
    \hline
    10000  & 10000 $\times$ 10000  & 300 \\
    \hline
\end{tabular}
\end{center}
\captionof{table}{Base de test 3}
\end{figure}

Base 3 : (pour cette base on souhaitait fixer edgeTreshold, cependant, un edgeTreshold de 50 pour un test de 10000 points dans un espace de 10000*10000 est trop long a générer) à la place nous avont choisi un compromis.

Une base 4 qui génère des points dans un espace beaucoup plus large que haut serait intéressante (a voir demain).

Tout ces tests seront effectués sur les deux version de l'algorithme (avec MIS1 et MIS2)

\paragraph{}
Résultats faire des graphes de taille du cds en fonction du nb de points et de temps en fonction du nombre de points pour MIS 1 et 2
4 graphes par bases (essayer de faire petit)

\begin{figure}
\begin{center}
\begin{tikzpicture}
\begin{axis}[%
  xlabel=Nombre de points,
  ylabel=Taille du CDS,
  xmin=0,
]
  \addplot[color=red,mark=x]  coordinates {
	(100, 35)
	(500, 169)
	(1000, 352)
	(5000, 377)
	(10000, 386)
};
  \addlegendentry{Base 1}
  \addplot[color=blue,mark=x]  coordinates {
	(100, 36)
	(500, 173)
	(1000, 351)
	(5000, 377)
	(10000, 384)
};
  \addlegendentry{Base 2}
  \addplot[color=green,mark=x] coordinates {
  	(100, 16)
	(500, 172)
	(1000, 352)
	(5000, 873)
	(10000, 1014)
};
  \addlegendentry{Base 3}
\end{axis}
\end{tikzpicture}
\end{center}
\captionof{figure}{Taille du CDS en fonction du nombre de points avec MIS1}
\end{figure}

BASE 1 ( 1 et 2 correspondent au MIS)
1 - 1 - 100 points - Average size : 35 points - Average time : 0.00129 s - Fails : 0
1 - 2 - 100 points - Average size : 37 points - Average time : 9.1E-4 s - Fails : 0
1 - 1 - 500 points - Average size : 169 points - Average time : 0.0046 s - Fails : 0
1 - 2 - 500 points - Average size : 174 points - Average time : 0.00425 s - Fails : 0
1 - 1 - 1000 points - Average size : 352 points - Average time : 0.01495 s - Fails : 0
1 - 2 - 1000 points - Average size : 364 points - Average time : 0.01403 s - Fails : 0
1 - 1 - 5000 points - Average size : 377 points - Average time : 0.29473 s - Fails : 0
1 - 2 - 5000 points - Average size : 394 points - Average time : 0.25969 s - Fails : 0
1 - 1 - 10000 points - Average size : 386 points - Average time : 1.16013 s - Fails : 0
1 - 2 - 10000 points - Average size : 402 points - Average time : 1.01815 s - Fails : 0

BASE 2 ( 1 et 2 correspondent au MIS)
2 - 1 - 100 points - Average size : 36 points - Average time : 2.0E-4 s - Fails : 0
2 - 2 - 100 points - Average size : 37 points - Average time : 2.4E-4 s - Fails : 0
2 - 1 - 500 points - Average size : 173 points - Average time : 0.00398 s - Fails : 0
2 - 2 - 500 points - Average size : 179 points - Average time : 0.00375 s - Fails : 0
2 - 1 - 1000 points - Average size : 351 points - Average time : 0.0149 s - Fails : 0
2 - 2 - 1000 points - Average size : 365 points - Average time : 0.01398 s - Fails : 0
2 - 1 - 5000 points - Average size : 377 points - Average time : 0.29543 s - Fails : 0
2 - 2 - 5000 points - Average size : 393 points - Average time : 0.26125 s - Fails : 0
2 - 1 - 10000 points - Average size : 384 points - Average time : 1.15217 s - Fails : 0
2 - 2 - 10000 points - Average size : 400 points - Average time : 1.01027 s - Fails : 0

BASE 3 ( 1 et 2 correspondent au MIS)
3 - 1 - 100 points - Average size : 16 points - Average time : 2.9E-4 s - Fails : 0
3 - 2 - 100 points - Average size : 18 points - Average time : 2.4E-4 s - Fails : 0
3 - 1 - 500 points - Average size : 172 points - Average time : 0.0039 s - Fails : 0
3 - 2 - 500 points - Average size : 180 points - Average time : 0.00367 s - Fails : 0
3 - 1 - 1000 points - Average size : 352 points - Average time : 0.01476 s - Fails : 0
3 - 2 - 1000 points - Average size : 365 points - Average time : 0.01395 s - Fails : 0
3 - 1 - 5000 points - Average size : 873 points - Average time : 0.32062 s - Fails : 0
3 - 2 - 5000 points - Average size : 907 points - Average time : 0.28485 s - Fails : 0
3 - 1 - 10000 points - Average size : 1014 points - Average time : 1.20708 s - Fails : 0
3 - 2 - 10000 points - Average size : 1054 points - Average time : 1.0546 s - Fails : 0
\newpage
\subsection{Discussion}

\paragraph{}
Grâce a cette analyse expérimentale, on peut déduire que notre première implémentation du MIS donne de moins bons résultats que la première implémentation. Un écart de plus de 30 points peut être observé sur le dernier test de la base 3.

Ce n'est pas étonnant car l'algorithme du MIS 2, bien qu'ayant la même complexité que celui du MIS 1, est plus lent d'un coefficient multiplicateur d'environ 10.

Une différence de temps de calcul pour une heuristique se traduit en une différence de qualité du résultat.

\paragraph{}
Au niveau des complexités, les graphiques ayant pour abscisse $nbpts \times \Delta$ avec $\Delta$ le degré maximum moyen rencontré dans les graphes d'un même ensemble de test, et comme ordonnée le temps de calcul d'un algorithme, permettent de confirmer notre étude théorique des complexités des algorithmes pour le MIS 1, MIS 2 et CDS.
Ces graphiques sont linéaires, ce qui correspond bien à une complexité de $O(n \times \Delta)$ avec $n = nombre de points du graphe$.

\paragraph{}
Le graphique présentant le temps d'exécution de l'algorithme final confirme bien la complexité quadratique due a la construction de la structure de graphe au début de l'algorithme. On est bien en $O(n^2)$.

\newpage
\section{Conclusion}
%
Finalement, il s'avère que l'algorithme Li et al. est très intéressant d'un point de vue temps d'exécution. En effet, l'algorithme glouton qui est proposé est de complexité quasi linéaire ce qui assure de trouver une solution en un temps séduisant. Cependant, bien que l'on puisse penser à une modeste qualité venant de la part d'un algorithme glouton, on nous assure un ratio de $(4.8+ln5)opt+1.2$ avec $opt$ la solution optimale, ce qui est remarquable. Nous avons donc ici, un compromis plus qu'intéressant entre temps de calcul et approximation de la meilleure solution.

On peut se poser une question sur cet algorithme, que les auteurs n'ont pas abordé : la pertinence d'un \textit{local searching}. En effet, il est coutume d'améliorer une solution donnée par une approche gloutonne au moyen de nombreuses micro optimisations que l'on appelle \textit{local searching}. On pourrait par exemple, prendre $n$ points de notre solution et tenter de les substituer par $n-1$ points tout en conservant les propriétés d'un ensemble dominant connexe.
\newpage
\nocite{*}
\bibliographystyle{plain}
\bibliography{bibliographie/ref}
\end{document}