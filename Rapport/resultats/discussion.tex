\newpage
\subsection{Discussion}

\paragraph{}
Grâce a cette analyse expérimentale, on peut déduire que notre première implémentation du MIS donne de moins bons résultats que la première implémentation. Un écart de plus de 30 points peut être observé sur le dernier test de la base 3.

Ce n'est pas étonnant car l'algorithme du MIS 2, bien qu'ayant la même complexité que celui du MIS 1, est plus lent d'un coefficient multiplicateur d'environ 10.

Une différence de temps de calcul pour une heuristique se traduit en une différence de qualité du résultat.

\paragraph{}
Au niveau des complexité, les graphiques ayant pour abscisse $nbpts \times \Delta$ avec $\Delta$ le degré maximum moyen rencontré dans les graphes d'un même ensemble de test, et comme ordonnée le temps de calcul d'un algorithme, permette de confirmer notre étude théorique des complexités des algorithmes pour le MIS 1, MIS 2 et CDS.
Ces graphiques sont linéaires, ce qui correspond bien à une complexité de $O(n \times \Delta)$ avec $n = nombre de points du graphe$.

\paragraph{}
Le graphique présentant le temps d'exécution de l'algorithme final confirme bien la complexité quadratique due a la construction de la structure de graphe au début de l'algorithme. On est bien en $O(n^2)$.